\documentclass[a0paper]{betterposter}

\usepackage[default]{lato}
\usepackage{inconsolata}

\usepackage{fontawesome}
\usepackage{graphicx}
\usepackage{rotating}
\usepackage{setspace}

\definecolor{mygrey}{RGB}{63, 63, 63}

% Title font
\renewcommand{\fontsizetitle}{\fontsize{60}{80} \selectfont}

% Main column font
\renewcommand{\fontsizemain}{\fontsize{130}{200} \selectfont}

%% Setting the width of columns
% Left column
\setlength{\leftbarwidth}{0.25\paperwidth}
% Right column
\setlength{\rightbarwidth}{0.25\paperwidth}

\begin{document}
\color{mygrey}

\betterposter{%

%%% MAIN COLUMN %%%
\maincolumn{%
    \textbf{Benchmark datasets}\\
    are \textbf{not the only option}\\
    when \textbf{understanding}\\
    algorithm \textbf{performance}
}{%
    \compactqrcode{img/paper_qr.eps}{%
        \textbf{Take a picture} to\\
        download the full paper
    }%
    \hfill%
    \includegraphics[width=.3\linewidth]{img/cu_logo}

    \vfill%
    \vspace{2em}\fontsize{40}{50}\selectfont\textbf{%
        \faGithub~\faTwitter\ \ daffidwilde
        \hfill%
        \faBook\ \ \texttt{edo.readthedocs.io}
        \hfill%
        \faNewspaperO\ \ \texttt{arxiv.org/abs/1907.13508}
    }
}

}{%

%%% LEFT COLUMN %%%
\vspace{-1em}\title{%
    Evolutionary dataset
    \\optimisation: learning
    \\algorithm quality
    \\through evolution
}
\author{%
    \vspace{-1.5ex}\textbf{Henry Wilde}, Vincent Knight,
    \\Jonathan Gillard
}

\vspace{-1.25em}\section{Motivating paradigm}
\vspace{-1.5em}\begin{center}
    \includegraphics[width=.9\linewidth]{tex/paradigm.pdf}
\end{center}

{\fontsize{26}{24}\selectfont\textcolor{mygrey}{%
    \textit{Left:} the proposed concept to understand algorithm performance by
    exploration of `good' (or bad) datasets. \textit{Right:} the established
    approach to algorithm selection by metric comparison.
}}

\vspace{-2.5em}\section{Using an evolutionary\\algorithm}

\vspace{-1em}\begin{minipage}{.04\linewidth}
    \
\end{minipage}
\begin{minipage}{.9\linewidth}
    \begin{itemize}
        \renewcommand\labelitemi{\faThumbsOUp~}
        \item Effective in complex domains
        \item Meaningful and adjustable design
        \item Transparent and rich solutions
    \end{itemize}

    \vspace{1ex}\begin{itemize}
        \renewcommand\labelitemi{\faHandORight~}
        \item Potential for suboptimal termination
        \item Finds the `easy' way out
    \end{itemize}

    \vspace{1ex}\begin{itemize}
        \renewcommand\labelitemi{\faThumbsODown~}
        \item Requires careful consideration of fitness
    \end{itemize}
\end{minipage}

\vspace{-.5em}
\section{Understanding performance of clustering algorithms}

\vspace{-1.5em}\begin{center}
    \begin{minipage}{.8\linewidth}
        \begin{tabular}{ccc}
            {} & Best & Worst\\
            \begin{turn}{90}\qquad\(k\)-means\end{turn} &
            \includegraphics[width=.45\linewidth]{img/kmeans_best.pdf} &
            \includegraphics[width=.45\linewidth]{img/kmeans_worst.pdf}\\
            \begin{turn}{90}\qquad DBSCAN\end{turn} &
            \includegraphics[width=.45\linewidth]{img/dbscan_best.pdf} &
            \includegraphics[width=.45\linewidth]{img/dbscan_worst.pdf}
        \end{tabular}
    \end{minipage}
\end{center}

{\fontsize{26}{24}\selectfont\textcolor{mygrey}{%
    Best and worst datasets from an attempt to maximise the silhouette of
    \(k\)-means (\textit{top}) whilst minimising that of DBSCAN
    (\textit{bottom}).
}}

}{%

%%% RIGHT COLUMN %%%

\vspace{-3em}\section{Inner mechanisms}
\fontsize{40}{50}\selectfont
Individual dataset representation

\vspace{.75ex}\begin{center}
    \includegraphics[width=.65\linewidth]{tex/individual.pdf}
\end{center}

Modified truncation selection

\vspace{.75ex}\begin{center}
    \includegraphics[width=.65\linewidth]{tex/selection.pdf}
\end{center}

Uniform-sampling crossover

\vspace{.75ex}\begin{center}
    \includegraphics[width=.8\linewidth]{tex/crossover.pdf}
\end{center}

Mutation process adapted for datasets

\vspace{.75ex}\begin{center}
    \includegraphics[width=.9\linewidth]{tex/mutation.pdf}
\end{center}
}

\end{document}
